\documentclass{article}
\usepackage[utf8, margin=1.3in, top=1in, bottom=1in]{geometry}

\title{Meeting Summons}
\author{Mikael Falkenberg Krog \and Magnus Gluppe \and Jakob Frantzvåg Karlsmoen \and Mikkel Aas}
\date{February - April 2020}

\usepackage{natbib}
\usepackage{graphicx}

\begin{document}

\maketitle
\tableofcontents
\newpage
\section{Meeting summons}
\subsection[Meeting summon: Project Group 2]{Meeting summon: Project Group 2\\ {\large 28.0.220202, 11:00, Atriet A-bygget NTNU Gjøvik}}

Magnus Gluppe is currently permanent meeting Chairman and minutes responsible.
\newline
\newline
\large Agenda 
\begin{itemize}
    \item Case no. 01/2020:  Opening meeting
    \item Case no. 02/2020:  Approval of agenda
    \item Case no. 03/2020:  Eating protocol during meetings
    \item Case no. 04/2020:  Startpoint and endpoint for a work sessions
    \item Case no. 05/2020:  Group versus individual work
    \item Case no. 06/2020:  Briefing everyone on individual work
\end{itemize}
\newline
\newline
Please contact me if you are unable to attend the meeting.

\newpage
\section{Minute}
\subsection{Minute from project meeting in Project Group 2}
\newline
\textbf{Time/location:} 28.02.2020, 11:00, Atriet A-bygget NTNU Gjøvik
\newline
\textbf{Present: }Mikkel Aas, Magnus Gluppe, Jakob Karlsmoen, Mikael Krog
\newline
\textbf{Absent:} No one
\newline
\textbf{Moderator:} Magnus Gluppe
 \newline \newline
\textbf{Case no 1/2020} \newline
Approved by acclamation.  \newline  \newline
\textbf{Case no 2/2020}  \newline
Anyone can eat during the meeting or work sessions, but we should avoid noisy food like chips. At important deadlines, the group can go out to eat. \newline  \newline
\textbf{Case no 3/2020}  \newline
Issue undecided, we like the approach of working until we are done with what we started. However, this method can have its disadvantages. Not everyone is productive at the same time, and might wish to end a session early and pick up the work later. If no one else does this, you can feel obligated to stay and produce a sub-par product.  \newline  \newline
\textbf{Case no 4/2020}  \newline
When we are further along with the project, it will become easier to work individually. The group still wishes to have joint work sessions, even if our tasks do not overlap. This ensures a certain amount of structure for everyone, and we can ask each other for help.  \newline  \newline
\textbf{Case no 5/2020}  \newline
In this part of the project, all the group members have cooperated on various documents and diagrams. So this point is not very relevant to this stage of the project. 


\begin{flushright}28.02.2020, Magnus Gluppe\end{flushright}

\end{document}
